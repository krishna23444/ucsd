% 
% installing.tex : part of the Mace toolkit for building distributed systems
% 
% Copyright (c) 2006, James W. Anderson, Charles Killian, Ryan Braud, Amin Vahdat
% All rights reserved.
% 
% Redistribution and use in source and binary forms, with or without
% modification, are permitted provided that the following conditions are met:
% 
%    * Redistributions of source code must retain the above copyright
%      notice, this list of conditions and the following disclaimer.
%    * Redistributions in binary form must reproduce the above copyright
%      notice, this list of conditions and the following disclaimer in
%      the documentation and/or other materials provided with the
%      distribution.
%    * Neither the names of Duke University nor The University of
%      California, San Diego, nor the names of the authors or contributors
%      may be used to endorse or promote products derived from
%      this software without specific prior written permission.
% 
% THIS SOFTWARE IS PROVIDED BY THE COPYRIGHT HOLDERS AND CONTRIBUTORS "AS IS"
% AND ANY EXPRESS OR IMPLIED WARRANTIES, INCLUDING, BUT NOT LIMITED TO, THE
% IMPLIED WARRANTIES OF MERCHANTABILITY AND FITNESS FOR A PARTICULAR PURPOSE ARE
% DISCLAIMED. IN NO EVENT SHALL THE COPYRIGHT OWNER OR CONTRIBUTORS BE LIABLE
% FOR ANY DIRECT, INDIRECT, INCIDENTAL, SPECIAL, EXEMPLARY, OR CONSEQUENTIAL
% DAMAGES (INCLUDING, BUT NOT LIMITED TO, PROCUREMENT OF SUBSTITUTE GOODS OR
% SERVICES; LOSS OF USE, DATA, OR PROFITS; OR BUSINESS INTERRUPTION) HOWEVER
% CAUSED AND ON ANY THEORY OF LIABILITY, WHETHER IN CONTRACT, STRICT LIABILITY,
% OR TORT (INCLUDING NEGLIGENCE OR OTHERWISE) ARISING IN ANY WAY OUT OF THE
% USE OF THIS SOFTWARE, EVEN IF ADVISED OF THE POSSIBILITY OF SUCH DAMAGE.
% 
% ----END-OF-LEGAL-STUFF----
\section{Getting and Installing Mace}
\label{sec:installing}

\subsection{Mace dependencies}

Mace requires the following system software packages:

\begin{description}

\item[gcc/g++] The GNU C and C++ compilers.  Mace officially supports GCC
  version 3.4, 4.0, and 4.1.  It has in the past worked with a few GCC
  version 3.3.X systems, but most frequently causes internal compiler
  segmentation faults.  Using at least version 3.4 is highly
  recommended.  This should also be accompanied with GNU Make.

\item[perl] Perl version 5.8 or greater.  Additionally, Class-MakeMethods and Parse-RecDescent modules are needed.
%TODO: I don't think people need to get these anymore, with mace-extras,
%right?

\item[system libraries] libpthread, libm, libcrypto, libstdc++, and libssl.

\item[libboost] Used in a few places for shared pointers and lexical casting.

\item[lgrind] For making the documentation from the source files, beautifying 
  the source code.  Also needs LaTeX.

\end{description}

Mace has been tested on Debian GNU/Linux and CentOS.

% XXX
% Chip, what else should we write here about platforms?


\subsection{Getting the Mace source code}
\label{sec:downloading}

Mace is Open Source Software which is be published under a BSD-style license.
The latest Mace source code release can be found at the following URL: \\

\href{http://mace.ucsd.edu/release}{http://mace.ucsd.edu/release} \\

Download this file and save it in the directory where you will be
doing your development.

To take full advantage of Mace, you should also download the Mace-extras
package (includes code licensed under the GPL), available at the same URL. \\

In particular, a sha1 implementation exists in the mace-extras package, though
it is released under a BSD compatible license.  Without this code, a simple
``hash'' will be computed instead.

\subsection{Installing Mace}
\label{sec:unpacking}

Currently, the Mace distribution does not require the installation of any
Mace-related files outside of the Mace source tree.  All development using Mace
can be done inside the source tree.

To unpack and build the Mace distribution with the mace-extras package, us the
following commands (latest is replaced by the version string from the version
downloaded):

\begin{screen}
\command{$ tar -xvzf mace-latest.tar.gz
$ tar -xvzf mace-extras-latest.tar.gz
$ mv mace-extras mace
$ cd mace
$ make
}
\end{screen}
%$

This will build the Mace libraries (described in
\S~\ref{sec:lib}), services
(\S~\ref{sec:packaged-services}), and applications
(\S~\ref{sec:applications}).

