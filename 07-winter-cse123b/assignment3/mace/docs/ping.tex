% 
% ping.tex : part of the Mace toolkit for building distributed systems
% 
% Copyright (c) 2006, James W. Anderson, Charles Killian, Ryan Braud, Amin Vahdat
% All rights reserved.
% 
% Redistribution and use in source and binary forms, with or without
% modification, are permitted provided that the following conditions are met:
% 
%    * Redistributions of source code must retain the above copyright
%      notice, this list of conditions and the following disclaimer.
%    * Redistributions in binary form must reproduce the above copyright
%      notice, this list of conditions and the following disclaimer in
%      the documentation and/or other materials provided with the
%      distribution.
%    * Neither the names of Duke University nor The University of
%      California, San Diego, nor the names of the authors or contributors
%      may be used to endorse or promote products derived from
%      this software without specific prior written permission.
% 
% THIS SOFTWARE IS PROVIDED BY THE COPYRIGHT HOLDERS AND CONTRIBUTORS "AS IS"
% AND ANY EXPRESS OR IMPLIED WARRANTIES, INCLUDING, BUT NOT LIMITED TO, THE
% IMPLIED WARRANTIES OF MERCHANTABILITY AND FITNESS FOR A PARTICULAR PURPOSE ARE
% DISCLAIMED. IN NO EVENT SHALL THE COPYRIGHT OWNER OR CONTRIBUTORS BE LIABLE
% FOR ANY DIRECT, INDIRECT, INCIDENTAL, SPECIAL, EXEMPLARY, OR CONSEQUENTIAL
% DAMAGES (INCLUDING, BUT NOT LIMITED TO, PROCUREMENT OF SUBSTITUTE GOODS OR
% SERVICES; LOSS OF USE, DATA, OR PROFITS; OR BUSINESS INTERRUPTION) HOWEVER
% CAUSED AND ON ANY THEORY OF LIABILITY, WHETHER IN CONTRACT, STRICT LIABILITY,
% OR TORT (INCLUDING NEGLIGENCE OR OTHERWISE) ARISING IN ANY WAY OUT OF THE
% USE OF THIS SOFTWARE, EVEN IF ADVISED OF THE POSSIBILITY OF SUCH DAMAGE.
% 
% ----END-OF-LEGAL-STUFF----
\section{Ping Revisited}
\label{sec:ping}

In this section, we revisit our Ping service, expanding it to make it
considerably more flexible and to demonstrate more aspects of Mace.


%%%%%%%%%%%%%%%%%%%%%%%%%%%%%%%%%%%%%%%%%%%%%%%%%%%%%%%%%%%%%%%%%%%%%%%%
% Full Ping
%%%%%%%%%%%%%%%%%%%%%%%%%%%%%%%%%%%%%%%%%%%%%%%%%%%%%%%%%%%%%%%%%%%%%%%%

\subsection{Revised Ping Implementation}
\label{sec:ping-implementation}

We are going to revise our Ping service to incorporate the following
changes:

\begin{enumerate}

\item Use UDP packets for sending messages, since the reliable
  byte-stream interface of TCP is unnecessary.

\item Use a run-time configurable ping interval and timeout.

\item Provide the ability to ping multiple hosts simultaneously.

\item Ping monitored hosts indefinitely (as opposed to sending a
  single message).

\item Support multiple registered handlers.

\end{enumerate}

The implementation for our revised service
\filename{services/Ping/Ping.m} follows:

\lgrindfile{Ping.m}

We discuss the details of the new implementation in
\S~\ref{sec:pingdetail}.  Next, we will present the revised Ping
application and usage.

Recompile your updated Ping service by running \command{make} in the
\directory{services} directory.

\subsection{Revised Ping Application}
\label{sec:ping-application}

We now present our revised ping application \filename{applications/ping/ping.cc} implementation:

\lgrindfile{ping.cc}

We have changed the implementation as follows:

\begin{enumerate}

\item Define \classname{PingLatencyHandler}, which reports the one-way
  latency for successful ping messages.

\item Remove calls to \function{exit()} in
  \classname{PingResponseHandler}.

\item In \function{main()}, instantiate a
  \classname{PingLatencyHandler}.

\item We now pass the values for \variablename{PING\_INTERVAL} and
  \variablename{PING\_TIMEOUT} as constructor parameters when we
  instantiate the Ping service object, overriding the defaults.

\item We register both handles with our Ping service instance using
  \function{registerHandler}, instead of
  \function{registerUniqueHandler}, and we assign the resulting
  registration uids to variables.

\item We loop over the command line arguments, and call
  \function{monitor}---once for each handler---with each host listed
  on the command line.

\end{enumerate}


The program takes zero or more hosts as command line arguments,
specified as either host names or IP addresses in the standard
numbers-and-dots notation.  If any hosts are specified, we have our
Ping service to monitor them.  If no hosts are specified, then the
program will simply respond to ping requests from other hosts.

To compile \filename{ping.cc}, add ``ping'' to the list of
applications specified by \variablename{APPS} in
\filename{applications/ping/Makefile}:

\verbatiminput{Makefile.ping}

You can then run make to build the new executable \filename{ping}.

Now try running the revised \command{ping} application.  Try passing
it multiple hosts on the command line.  Begin monitoring a host; then
kill the \command{ping} process, wait a few seconds, and restart it on
the monitored host; watch the output on the monitoring host.

%% \subsection{Running Revised Ping}
%% \label{sec:running-ping}

%% Copy the \filename{ping} executable to another machine.  Assuming your
%% two machines are named \replaceable{alexander} and \replaceable{philip}, run the
%% following commands.

%% \noindent
%% On \replaceable{philip}:

%% \begin{screen}
%% \command{philip$ ./ping}
%% \end{screen}
%% %$

%% \noindent
%% On \replaceable{alexander}:

%% \begin{screen}
%% \command{alexander$ ./ping philip
%% }
%% IPV4/philip(192.168.0.2) responded in 119001 usec
%% IPV4/philip(192.168.0.2) responded in 331 usec
%% IPV4/philip(192.168.0.2) responded in 344 usec
%% ...
%% \end{screen}
%% %$

%% \noindent
%% Now kill the ping process on \replaceable{philip}.  Then
%% \replaceable{alexander} will report:

%% \small
%% \begin{screen}
%% ...
%% IPV4/philip(192.168.0.2) responded in 344 usec
%% did not receive response from IPV4/philip(192.168.0.2) pinged at Mon May  9 22:03:04 2005
%% did not receive response from IPV4/philip(192.168.0.2) pinged at Mon May  9 22:03:06 2005
%% did not receive response from IPV4/philip(192.168.0.2) pinged at Mon May  9 22:03:08 2005
%% ...
%% \end{screen}
%% \normalsize

%% Restarting ping on \replaceable{philip} will cause \replaceable{alexander} to
%% resume reporting the latency.  You can further experiment with ping by
%% having \replaceable{philip} and \replaceable{alexander} ping each other at the
%% same time, and by pinging multiple hosts at the same time (i.e.:
%% \command{\$~./ping~philip~illyria~thrace}).

