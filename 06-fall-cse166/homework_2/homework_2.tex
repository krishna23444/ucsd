\documentclass[10pt,letterpaper,oneside,onecolumn,leqno,fleqn]{article}

\usepackage{graphicx}

\author{Nitay Joffe}
\title{CSE 166 Homework 2}
\date{10/12/2006}

\begin{document}
  \maketitle

  \textbf{Written exercises}
  \begin{enumerate}
    \item
      GW, Problem 10.5. \\
      Suppose we have used the edge model shown, instead of the ramp model of
      an edge shown in Fig10.6. Sketch the Gradient and Laplacian of each
      profile. \\

    \pagebreak
    \item
      GW, Problem 10.8. \\
      $$ \left[ \begin{array}{rrr}
      -1 & 0 & 1
      \end{array}\right] *
      \left[ \begin{array}{r}
      1 \\
      2 \\
      1
      \end{array} \right] =
      \left[ \begin{array}{rrr}
      -1 & 0 & 1 \\
      -2 & 0 & 2 \\
      -1 & 0 & 1
      \end{array}\right] $$ \\
      $$ \left[ \begin{array}{rrr}
      1 & 2 & 1
      \end{array}\right] *
      \left[ \begin{array}{r}
      -1 \\
      0  \\
      1
      \end{array} \right] =
      \left[ \begin{array}{rrr}
      -1 & -2 & -1 \\
       0 & 0  &  0 \\
      1  & 2  &  1
      \end{array}\right] $$
    
    \bigskip \bigskip
    \item
      Compute $ \nabla^2f $ for the function $ f(x,y) =
      e^{-(x^2+y^2)/2\sigma^2} $. Show your work.
      
      $$ f(x,y) = e^{-(x^2+y^2)/2\sigma^2} =
      e^{-x^2/2\sigma^2}e^{-y^2/2\sigma^2} $$
      
      $$ \frac{\partial f}{\partial x} = \frac{-x}{\sigma^2}
      e^{-x^2/2\sigma^2}e^{-y^2/2\sigma^2} $$
      $$ \frac{\partial^2 f}{\partial x^2} = \frac{-1}{\sigma^2}
      e^{-x^2/2\sigma^2}e^{-y^2/2\sigma^2} + \frac{x^2}{\sigma^4} 
      e^{-x^2/2\sigma^2}e^{-y^2/2\sigma^2} =
      \left(-1 + \frac{x^2}{\sigma^2} \right)
      \frac{1}{\sigma^2}e^{-x^2/2\sigma^2}e^{-y^2/2\sigma^2} $$

      $$ \frac{\partial f}{\partial y} = \frac{-y}{\sigma^2}
      e^{-x,^2/2\sigma^2}e^{-y^2/2\sigma^2} $$
      $$ \frac{\partial^2 f}{\partial y^2} = \frac{-1}{\sigma^2}
      e^{-x^2/2\sigma^2}e^{-y^2/2\sigma^2} + \frac{y^2}{\sigma^4} 
      e^{-x^2/2\sigma^2}e^{-y^2/2\sigma^2}  =
      \left(-1 + \frac{y^2}{\sigma^2} \right)
      \frac{1}{\sigma^2}e^{-x^2/2\sigma^2}e^{-y^2/2\sigma^2} $$

      $$ \nabla^2f = \frac{\partial^2 f}{\partial x^2} +
      \frac{\partial^2 f}{\partial y^2} =
      \left(-1 + \frac{x^2}{\sigma^2} \right)
      \frac{1}{\sigma^2}e^{-x^2/2\sigma^2}e^{-y^2/2\sigma^2} +
      \left(-1 + \frac{y^2}{\sigma^2} \right)
      \frac{1}{\sigma^2}e^{-x^2/2\sigma^2}e^{-y^2/2\sigma^2} $$ \\
      $$ \nabla^2f =
      \left(-2 + \frac{x^2}{\sigma^2} + \frac{y^2}{\sigma^2} \right)
      \frac{1}{\sigma^2}e^{-x^2/2\sigma^2}e^{-y^2/2\sigma^2} $$

    \pagebreak
    \item
      GW, Problem 10.11.

    \pagebreak
    \item
      GW, Problem 4.1. \\
      \linebreak
      Substituting $ f(x) $ [Eq. (4.2-6)] into $ F(u) $ [Eq. (4.2-5)]:
      \begin{eqnarray*}
        F(u) &=& \frac{1}{M}\sum_{x=0}^{M-1} f(x)e^{-j2\pi ux/M} \\
             &=& \frac{1}{M}\sum_{x=0}^{M-1}\left[ \sum_{r=0}^{M-1} 
                 F(r) e^{j2\pi rx/M} \right] e^{-j2\pi ux/M} \\
             &=& \frac{1}{M}\sum_{r=0}^{M-1} F(r) \sum_{x=0}^{M-1} 
                 e^{j2\pi rx/M} e^{-j2\pi ux/M} \\
             &=& \frac{1}{M}F(u)[M] \\
             &=& F(u)
      \end{eqnarray*} \\
      \linebreak
      Substituting $ F(u) $ [Eq. (4.2-5)] into $ f(x) $ [Eq. (4.2-6)]:
      \begin{eqnarray*}
        f(x) &=& \sum_{u=0}^{M-1} F(u)e^{j2\pi ux/M} \\
             &=& \sum_{u=0}^{M-1} \left[ \frac{1}{M}\sum_{r=0}^{M-1}
                 f(r)e^{-j2\pi ur/M} \right] e^{j2\pi ux/M} \\
             &=& \frac{1}{M}\sum_{r=0}^{M-1} f(r) \sum_{u=0}^{M-1} 
                 e^{-j2\pi ur/M} e^{j2\pi ux/M} \\
             &=& \frac{1}{M}f(x)[M] \\
             &=& f(x)
      \end{eqnarray*}

    \pagebreak
    \item
      GW, Problem 4.4.
      \begin{eqnarray*}
               &&  let \quad w^2 = u^2 + v^2 \\
        H(w)   &=& e^{-w^2/2\sigma^2} \\
        h(z)   &=& \int_{-\infty}^{\infty}H(w)e^{j2\pi wz}dw \\
               &=& \int_{-\infty}^{\infty}e^{-w^2/2\sigma^2}e^{j2\pi wz}dw \\
               &=& \int_{-\infty}^{\infty}e^{-\frac{1}{2\sigma^2}\left[
                   w^2-j4\pi\sigma^2wz\right]}dw \\
        e^{-\frac{(2\pi)^2z^2\sigma^2}{2}}e^{\frac{(2\pi)^2z^2\sigma^2}{2}} &=& 1 \\
        h(z)   &=& e^{-\frac{(2\pi)^2z^2\sigma^2}{2}} \int_{-\infty}^{\infty}
                   e^{-\frac{1}{2\sigma^2}\left[w^2-j4\pi\sigma^2wz-
                   (2\pi)^2\sigma^4z^2\right]}dw \\
               &=& e^{-\frac{(2\pi)^2z^2\sigma^2}{2}} \int_{-\infty}^{\infty}
                   e^{-\frac{1}{2\sigma^2}}\left[w-j2\pi\sigma^2z\right]^2dw \\
               &&  let \quad r = w - j2\pi\sigma^2z \quad \to \quad dr = dw \\
        h(z)   &=& e^{-\frac{(2\pi)^2z^2\sigma^2}{2}} \int_{-\infty}^{\infty}
                   e^{-\frac{r^2}{2\sigma^2}} dr \\
               &=& \sqrt{2\pi}\sigma e^{-\frac{(2\pi)^2z^2\sigma^2}{2}}
                   \left[\frac{1}{\sqrt{2\pi}\sigma}\int_{-\infty}^{\infty}
                   e^{-\frac{r^2}{2\sigma^2}}dr \right] \\
               &=& \sqrt{2\pi}\sigma e^{-\frac{(2\pi)^2z^2\sigma^2}{2}} \\
        h(x,y) &=& \sqrt{2\pi}\sigma e^{-2\pi^2\sigma^2(x^2+y^2)}
      \end{eqnarray*}

    \pagebreak
    \item
      GW, Problem 4.6. Explain how this relates to the Matlab function
      \texttt{fftshift}.
      \begin{enumerate}
        \item
          Prove the validity of Eq.(4.2-21).
          \begin{eqnarray*}
            \Im\left[ f(x,y)(-1)^{x+y} \right]
            &=& \Im\left[ f(x,y)e^{j\pi(x+y)} \right] \\
            \Im\left[ f(x,y)e^{j\pi(x+y)} \right]
            &=& \frac{1}{MN} \sum_{x=0}^{M-1} \sum_{y=0}^{N-1} \left[
                f(x,y)e^{j\pi(x+y)} \right] e^{-j2\pi(ux/M+vy/N)} \\
            &=& \frac{1}{MN} \sum_{x=0}^{M-1} \sum_{y=0}^{N-1} \left[f(x,y)
                e^{-j2\pi \left( -\frac{xM}{2M}-\frac{yN}{2N} \right)} \right]
                e^{-j2\pi(ux/M+vy/N)} \\
            &=& \frac{1}{MN} \sum_{x=0}^{M-1} \sum_{y=0}^{N-1} f(x,y)
                e^{-j2\pi \left(x\left[u-\frac{M}{2}\right]/M +
                y\left[v-\frac{N}{2}\right]/N\right)} \\
            &=& F(u-M/2, v-N/2)
          \end{eqnarray*}

        \bigskip \bigskip
        \item
          Prove the validity of Eqs.(4.6-1) and (4.6-2).
          \begin{eqnarray*}
            \Im\left[f(x,y)e^{j2\pi(u_0x/M+v_0y/N)}\right]
            &=& \frac{1}{MN} \sum_{x=0}^{M-1} \sum_{y=0}^{N-1} \left[
                f(x,y)e^{j2\pi(u_0x/M+v_0y/N)} \right] \\
            &&  \qquad \qquad \qquad \quad e^{-j2\pi(ux/M+vy/N)} \\
            &=& \frac{1}{MN} \sum_{x=0}^{M-1} \sum_{y=0}^{N-1} \left[ f(x,y)
                e^{-j2\pi(-u_0x/M-v_0y/N)} \right] \\
            &&  \qquad \qquad \qquad \quad e^{-j2\pi(ux/M+vy/N)} \\
            &=& \frac{1}{MN} \sum_{x=0}^{M-1} \sum_{y=0}^{N-1} f(x,y)
                e^{-j2\pi([u-u_0]x/M+[v-v_0]y/N)} \\
            &=& F(u-u_0, v-v_0)
          \end{eqnarray*}
          \begin{eqnarray*}
            \Im\left[f(x-x_0, y-y_0)\right]
            &=& \frac{1}{MN} \sum_{x=0}^{M-1} \sum_{y=0}^{N-1} \left[
                f(x-x_0,y-y_0)e^{-j2\pi(ux/M+vy/N)} \right] \\
            &&  let \quad A = x-x_0 \quad and \quad B = y - y_0 \\
            &=& \frac{1}{MN} \sum_{x=0}^{M-1} \sum_{y=0}^{N-1} \left[
                f(A,B) e^{-j2\pi(u[A+x_0]/M+v[B+y_0]/N)} \right] \\
            &=& \frac{1}{MN} \sum_{x=0}^{M-1} \sum_{y=0}^{N-1} \left[
                f(A,B) e^{-j2\pi(uA/M+vB/N)} \right]
                e^{-j2\pi(ux_0/M+vy_0/N)} \\
            &&  rename \quad A \to x \quad and \quad B \to y \\
            &=& \frac{1}{MN} \sum_{x=0}^{M-1} \sum_{y=0}^{N-1} \left[
                f(x,y) e^{-j2\pi(ux/M+vy/N)} \right]
                e^{-j2\pi(ux_0/M+vy_0/N)} \\
            &=& F(u,v) e^{-j2\pi(ux_0/M+vy_0/N)}
          \end{eqnarray*}
    \end{enumerate}
  \end{enumerate}

  \pagebreak 
  \textbf{Matlab exercises}
  \begin{enumerate}
    \item[2]
      Canny Edge Detection
      \begin{enumerate}
        \item
        Examine the \texttt{gradient} function and explain how it is
        implemented in terms of convolution. (The commands \texttt{which} and
        \texttt{type} may be useful for this purpose. Alternatively, you can
        apply \texttt{gradient} to an impulse and inspect the result.) \\
        \linebreak \linebreak
        The \texttt{gradient} function applies a centered first order
        difference kernel across the x and y directions. In other words, it
        convolves the function with $ \begin{array}{rrr} [0.5 & 0 & -0.5]
        \end{array} $ followed by $ \left[ \begin{array}{c} 0.5 \\ 0 \\ -0.5
        \end{array} \right]/2 $. \\ \\

        \bigskip \bigskip
        \item[(c)]
          Run your edge detector on Figure 10.4(a) using $ \tau = 5 $ and the
          following three values for $ \sigma^2 $: 0.5, 1, and 3. (Note: to
          save ink, invert E before printing it.) Discuss how the choice of
          $ \sigma^2 $ effects the results. \\
          \linebreak \linebreak
          The larger the $ \sigma^2 $ that's used, the bigger the smoothing
          kernel is. When the image is smoothed with larger kernels, there is
          more blurring which causes the edges to smoothen out. This means that
          less faulty edges due to noise will be detected, but it also makes it
          more likely to miss important edges.
      \end{enumerate}

    \pagebreak
  \end{enumerate}
\end{document}
