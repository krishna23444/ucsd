\documentclass{article}
\usepackage{bbm}

\marginparwidth  0pt
\oddsidemargin   0pt
\evensidemargin  0pt
\marginparsep    0pt
\topmargin       0pt
\textwidth     6.0in
\textheight    8.5in

\title{CSE166 Image Processing Homework 3}
\author{Nitay Joffe}
\date{10/19/2006}

\begin{document}
  \maketitle

  \noindent{\bf Written exercises}
  \begin{enumerate}
    \item \textit{GW, Problem 4.2.\\
    Show that the Fourier transform and its inverse are linear processes.}
    \begin{eqnarray*}
      af_1(x)+bf_2(x) &\longleftrightarrow & F(u)\\
      F(u) &=& \frac{1}{M}\sum_{x=0}^{M-1}[af_1(x)+bf_2(x)]e^{-j2\pi ux/M}\\
           &=& a\frac{1}{M}\sum_{x=0}^{M-1}f_1(x)e^{-j2\pi ux/M} +
               b\frac{1}{M}\sum_{x=0}^{M-1}f_2(x)e^{-j2\pi ux/M}\\
           &=& aF_1(u) + bF_2(u)
    \end{eqnarray*}
    \begin{eqnarray*}
      f(x) &\longleftrightarrow & aF_1(u)+bF_2(u)\\
      f(x) &=& \sum_{u=0}^{M-1}[aF_1(u)+bF_2(u)]e^{j2\pi ux/M}\\
           &=& a\sum_{u=0}^{M-1}F_1(u)e^{j2\pi ux/M} +
               b\sum_{u=0}^{M-1}F_2(u)e^{j2\pi ux/M}\\
           &=& af_1(x) + bf_2(x)
    \end{eqnarray*}

    \item \textit{GW, Problem 4.7.\\
    What is the source of the nearly periodic bright points in the horizontal
    axis of the spectrum in Fig4.11(b)?}\\
    \\
    The bright points are the high values of the sinc function, which is the
    result of running the discrete fourier transform on a box. The original
    image has many boxes and rectangles, so it is likely caused by them.\\

    \item \textit{GW, Problem 4.8.\\
    Each of the spatial filters shown in Fig4.23 has a strong spike at the
    origin. Explain the source of these spikes.}\\
    \\
    The spikes are caused by the $1-...$ portion of the equations that all the
    high pass filters have. The inverse discrete fourier transform of 1 is
    a delta function.\\

    \pagebreak
    \item \textit{GW, Problem 4.10.\\
    Show that if a filter transfer function H(u,v) is real and symmetric, then
    the corresponding spatial domain filter h(x,y) also is real and symmetric.}\\
    $$ H(u,v) = H(-u,v) = H(u,-v) $$
    $$ h(-x,y) = \sum_{u=0}^{M-1}\sum_{v=0}^{N-1} H(u,v) e^{-j2\pi ux/M}
                                                         e^{j2\pi vy/M} $$
    $$ let \;\; a = -u, u = -a $$
    $$ h(-x,y) = \sum_{a=0}^{M-1}\sum_{v=0}^{N-1} H(-a,v) e^{j2\pi ax/M}
                                                          e^{j2\pi vy/M} $$
    $$ h(-x,y) = \sum_{a=0}^{M-1}\sum_{v=0}^{N-1} H(a,v) e^{j2\pi ax/M}
                                                         e^{j2\pi vy/M} $$
    $$ h(-x,y) = h(x,y) $$\\
    The argument for the y case is the same, just using different variable.\\

    \item \textit{GW, Problem 4.15.\\
    The basic approach used to approximate a discrete derivative (Section 3.7)
    involves taking differences of the form $ f(x+1,y)-f(x,y) $.\\
      (a) Obtain the filter transfer function, H(u,v), for performing the
          equivalent process in the frequency domain.}\\
    \begin{eqnarray*}
      f(x,y) &\longleftrightarrow & F(u,v)\\
      f(x+1,y) &\longleftrightarrow & F(u,v)e^{j2\pi u/M}\\
      f(x,y+1) &\longleftrightarrow & F(u,v)e^{j2\pi v/N}\\
      g(x,y) &=& f(x+1,y) - f(x,y) + f(x,y+1) - f(x,y)\\
      g(x,y) &\longleftrightarrow & G(u,v)\\
      G(u,v) &\longleftrightarrow & F(u,v)e^{j2\pi u/M}-F(u,v) +
                                    F(u,v)e^{j2\pi v/N}-F(u,v)\\
             &=& \left[e^{j2\pi u/M}-1\right]F(u,v) +
                 \left[e^{j2\pi v/N}-1\right]F(u,v)\\
    \end{eqnarray*}
      \textit{(b) Show that H(u,v) is a highpass filter.}\\
      \linebreak
      When u or v are zero, the respective exponential of that part of the
      filter goes to one, which cancels out the -1 term. This shows that low
      frequencies are attenuated by this filter. As u or v increase to
      infinity, the exponential term grows unbounbed, which makes the overall
      term also grow since the -1 constant does not change. This shows that
      high frequencies are passed through. This behavior is exactly that of
      a high pass filter.\\

    \pagebreak
    \item \textit{GW, Problem 4.18.\\
    Can you think of a way to use the Fourier transform to compute (or
    partially compute) the magnitude of the gradient for use in image
    differentiation (see Section 3.7.3)? If your answer is yes, give a method
    to do it. If your answer is no, explain why.}\\
    \linebreak
    No. Computing the magnitude of the gradient of an image requires
    calculating squares and square roots. These calculations are not linear,
    and therefore a Fourier transform, which is a linear process, cannot be
    used to compute them.\\
  \end{enumerate}

  \noindent{\bf Matlab exercises}
  \begin{enumerate}
    \item \textit{Filtered Noise}
    \begin{enumerate}
      \item[(b)] \textit{
      Display the filtered image along with the
      original noise image.  Quite remarkably, the filtered image should
      look like a ``natural'' texture, such as clouds or terrain.  What does
      this suggest about the statistics of natural images vs.\ that of
      images of manmade objects?}\\
      \\
      The fact that the filtered normally distributed noise images look like
      suggests that the objects and behavior humans see and experience in our
      surroundings is caused by random processes, while the structures that
      people create tend to have very regular structure. Images of buildings
      have and cars have a defined shape, while images of natural things have
      much more random variation.
    \end{enumerate}
  \end{enumerate}
\end{document}
