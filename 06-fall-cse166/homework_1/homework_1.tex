\documentclass[9pt]{article}

%\usepackage{amsmath}
%\usepackage{amssymb}
%\usepackage{amsthm}
%\usepackage{array}
%\usepackage{color}
\usepackage{graphicx}
%\usepackage{xy}

\pdfpagewidth 8.5in
\pdfpageheight 11in

\topmargin 0in
\headheight 0in
\headsep 0in
\textheight 7.7in
\textwidth 6.5in
\oddsidemargin 0in
\evensidemargin 0in
\headheight 77pt
\headsep 0.25in

\title{CSE 166 Homework 0}
\author{Nitay Joffe}
\date{10/02/2006}

\begin{document}
  \maketitle
  \textbf{WRITTEN EXERCISES}
  \begin{enumerate}
    \item
      GW, Problem 3.17
      \begin{enumerate}
        \item
          Since the kernel we are using is a box-filter, that is a square
          matrix with all coefficients equal to one, no multiplication needs to
          be done. All the computations involved are additions of pixel values
          in the image.
          Initially, we compute the $ n \times n $ image kernel at the top
          left of the image. This takes $ n^{2} - 1 $ additions. When we move
          the kernel to the right by one pixel we are adding an $ n \times 1 $
          column of pixels on the right side while removing the $ n \times 1 $
          column of pixels on the left. In particular, the first time we do
          this, we are adding the $ (n+1)th $ column while removed the $ first
          $ column. Using this observation we can compute the new sum by simply
          adding the column on the right side that was shifted in to the kernel
          and subtracting the column on the left side that was shifted out of
          the kernel. Mathematically speaking, we have $ Pixel_{new} =
          Pixel_{old} - Column_{shifted\_out} + Column_{shifted\_in} $.
          Computing the sum of an $ n \times 1 $ column of values takes $ n-1 $
          additions.
          We store the result of previous computations so that we
          don't have to recompute $ C_{shifted\_out} $ each time. The storage
          required is fairly minimal since all we need is enough space to store
          the n column sums.
          Computing the value of the new pixel requires
          $ n - 1 $ additions for the new column plus $ 1 $ subtraction of the
          shifted out column and $ 1 $ addition for the shifted in column,
          which equals $ n + 1 $ addition operations.
          Moving the kernel down one pixel works in a similar
          fashion using the shifted out row on top and the shifted in row on
          the bottom. \\
        \item
          The brute-force implementation uses $ n^{2} - 1 $ additions for each
          pixel. My algorithm uses $ n^{2} - 1 $ additions for the first pixel,
          and $ n + 1 $ additions for every pixel thereafter. Therefore, the
          \textit{computational advantange}, that is the ratio of the number of
          computations performed by the brute-force implementation to my
          algorithm, is equal to $ \displaystyle \frac{n^{2} - 1}{n + 1}
          = n - 1 $. \\
      \end{enumerate}
    \item
      GW, Problem 3.18 \\
      With each repetition of the 3x3 lowpass spatial filter, the pixels in the
      image will become more and more blurred. Eventually, textures and shapes
      will fade out and there won't be any visible shapes, just one big blur of
      colors. Each application of the lowpass filter brings the pixel values
      closer together by blurring them. That means there is less
      differentiation between the pixels, which makes the next application of
      the lowpass filter less effective. \\
    \item
      GW, Problem 3.22 \\
      The reason the bars in (b) have merged is because the size of the mask is
      exactly the length of one column plus the space in between the columns.
      When this is the case, no matter where the center of the mask is placed
      along the lines, the value will always be the same. It all comes down to
      how the frequency of repetition in the bar pattern relates to the size of
      the mask being used. When there is a periodic signal (like this bar
      pattern) and a cumulative function over an area that is some multiple of
      the frequency of the signal is being computed, the location of the center
      of the area does not matter because the same periodic data will be
      captured. In this particular case, the mask in (b) is of length 25, which
      means the period of the bar pattern must be either 5 or 25. However, in
      (c), when a mask of 45 is used, the blurring is not the same throughout,
      so 5 cannot be the period. That leaves 25 as the period of the bar
      pattern. In (a), a mask smaller than the period length is used, so the
      offsets have an effect and distinctions occur.
    \item
      \begin{enumerate}
        \item
          $ g(x) = e^{f(x)} $
          \begin{enumerate}
          \item
            \textbf{NOT linear}
            \begin{eqnarray*}
              r(x) &=& \alpha f_1(x) + \beta f_2(x) \\
              H[r(x)] &=& e^{r(x)} \\
                      &=& e^{\alpha f_{1}(x) + \beta f_{2}(x)} \\
                      &=& e^{\alpha f_{1}(x)} e^{\beta f_{2}(x)} \\
              \\
              \alpha H[f_1(x)] + \beta H[f_2(x)]
                      &=& \alpha e^{f_1(x)} + \beta e^{f_2(x)} \\
            \end{eqnarray*}
          \item
            \textbf{spatially invariant}
            \begin{eqnarray*}
              r(x) &=& f(x + x_0) \\
              H[r(x)] &=& e^{r(x)} \\
                      &=& e^{f(x + x_0)} \\
              \\
              g(x + x_0) &=& e^{f(x + x_0)} \\
            \end{eqnarray*}
          \end{enumerate}
        \item
          $ g(x) = f(x)f(x-1) $
          \begin{enumerate}
          \item
            \textbf{NOT linear}
            \begin{eqnarray*}
              r(x) &=& \alpha f_1(x) + \beta f_2(x) \\
              H[r(x)] &=& r(x)r(x-1) \\
                      &=& [\alpha f_1(x) + \beta f_2(x)]
                          [\alpha f_1(x-1) + \beta f_2(x-1)] \\
                      &=& \alpha^2 f_1(x) f_1(x-1) + \beta^2 f_2(x) f_2(x-1) \\
                      &&  + \alpha \beta [f_1(x) f_2(x-1) + f_2(x) f_1(x-1)] \\
              \\
              \alpha H[f_1(x)] + \beta H[f_2(x)]
                      &=& \alpha f_1(x) f_1(x-1) + \beta f_2(x) f_2(x-1) \\
            \end{eqnarray*}
          \item
            \textbf{spatially invariant}
            \begin{eqnarray*}
              r(x) &=& f(x + x_0) \\
              H[r(x)] &=& r(x)r(x-1) \\
                      &=& f(x + x_0)f(x + x_0 - 1) \\
              \\
              g(x + x_0) &=& f(x + x_0)f(x + x_0 - 1) \\
            \end{eqnarray*}
          \end{enumerate}
        \item
          $ g(x) = f(x) - f(x-1) $
          \begin{enumerate}
          \item
            \textbf{linear}
            \begin{eqnarray*}
              r(x) &=& \alpha f_1(x) + \beta f_2(x) \\
              H[r(x)] &=& r(x) - r(x-1) \\
                      &=& \alpha f_1(x) + \beta f_2(x) - \alpha f_1(x - 1)
                          - \beta f_2(x - 1) \\
                      &=& \alpha (f_1(x)-f_1(x-1)) + \beta (f_2(x)-f_2(x-1)) \\
              \\
              \alpha H[f_1(x)] + \beta H[f_2(x)]
                      &=& \alpha (f_1(x)-f_1(x-1)) + \beta (f_2(x)-f_2(x-1)) \\
            \end{eqnarray*}
          \item
            \textbf{spatially invariant}
            \begin{eqnarray*}
              r(x) &=& f(x + x_0) \\
              H[r(x)] &=& r(x) - r(x - 1) \\
                      &=& f(x + x_0) - f(x + x_0 - 1) \\
              \\
              g(x + x_0) &=& f(x + x_0) - f(x + x_0 - 1) \\
            \end{eqnarray*}
          \end{enumerate}
        \item
          $ g(x) = f(x-2) - 2f(x-17) $
          \begin{enumerate}
          \item
            \textbf{linear}
            \begin{eqnarray*}
              r(x) &=& \alpha f_1(x) + \beta f_2(x) \\
              H[r(x)] &=& r(x-2) - 2r(x-17) \\
                      &=& \alpha f_1(x-2) + \beta f_2(x-2) -
                          2[\alpha f_1(x-17) + \beta f_2(x-17)] \\
                      &=& \alpha [f_1(x-2) - 2f_1(x-17)] +
                          \beta [f_2(x-2) - 2f_2(x-17)] \\
              \\
              \alpha H[f_1(x)] + \beta H[f_2(x)]
                      &=& \alpha [f_1(x-2) - 2f_1(x-17)] +
                          \beta [f_2(x-2) - 2f_2(x-17)] \\
            \end{eqnarray*}
          \item
            \textbf{spatially invariant}
            \begin{eqnarray*}
              r(x) &=& f(x + x_0) \\
              H[r(x)] &=& r(x + x_0 - 2) - 2r(x + x_0 - 17) \\
              \\
              g(x + x_0) &=& f(x + x_0 - 2) - 2f(x + x_0 - 17) \\
            \end{eqnarray*}
          \end{enumerate}
        \item
          $ g(x) = \sin(6x) f(x) $
          \begin{enumerate}
          \item
            \textbf{linear}
            \begin{eqnarray*}
              r(x) &=& \alpha f_1(x) + \beta f_2(x) \\
              H[r(x)] &=& \sin(6x) r(x) \\
                      &=& \sin(6x) [\alpha f_1(x) + \beta f_2(x)] \\
                      &=& \alpha \sin(6x) f_1(x) + \beta \sin(6x) f_2(x) \\
              \\
              \alpha H[f_1(x)] + \beta H[f_2(x)]
                      &=& \alpha \sin(6x) f_1(x) + \beta \sin(6x) f_2(x) \\
            \end{eqnarray*}
          \item
            \textbf{NOT spatially invariant}
            \begin{eqnarray*}
              r(x) &=& f(x + x_0) \\
              H[r(x)] &=& \sin(6x) r(x) \\
                      &=& \sin(6x) f(x + x_0) \\
              \\
              g(x + x_0) &=& \sin(6(x + x_0)) f(x + x_0) \\
            \end{eqnarray*}
          \end{enumerate}
        \item
          $ g(x) = \sum_{k = x - 2}^{x + 4} f(x) $
          \begin{enumerate}
          \item
            \textbf{linear}
            \begin{eqnarray*}
              r(x) &=& \alpha f_1(x) + \beta f_2(x) \\
              H[r(x)] &=& \sum_{k=x-2}^{x+4} r(x) \\
                      &=& \sum_{k=x-2}^{x+4} [\alpha f_1(x) + \beta f_2(x)] \\
                      &=& \alpha \sum_{k=x-2}^{x+4} f_1(x) +
                          \beta \sum_{k=x-2}^{x+4} f_2(x) \\
              \\
              \alpha H[f_1(x)] + \beta H[f_2(x)]
                      &=& \alpha \sum_{k=x-2}^{x+4} f_1(x) +
                          \beta \sum_{k=x-2}^{x+4} f_2(x)
            \end{eqnarray*}
          \item
            \textbf{spatially invariant}
            \begin{eqnarray*}
              r(x) &=& f(x + x_0) \\
              H[r(x)] &=& \sum_{k=x-2}^{x+4} r(x) \\
                      &=& \sum_{k=x-2}^{x+4} f(x + x_0) \\
              \\
              g(x + x_0) &=& \sum_{k=x-2}^{x+4} f(x + x_0)
            \end{eqnarray*}
          \end{enumerate}
        \item
          $ g(x) = xf(x) $
          \begin{enumerate}
          \item
            \textbf{linear}
            \begin{eqnarray*}
              r(x) &=& \alpha f_1(x) + \beta f_2(x) \\
              H[r(x)] &=& x r(x) \\
                      &=& \alpha x f_1(x) + \beta x f_2(x) \\
              \\
              \alpha H[f_1(x)] + \beta H[f_2(x)]
                      &=& \alpha x f_1(x) + \beta x f_2(x)
            \end{eqnarray*}
          \item
            \textbf{NOT spatially invariant}
            \begin{eqnarray*}
              r(x) &=& f(x + x_0) \\
              H[r(x)] &=& x r(x) \\
                      &=& x f(x + x0) \\
              \\
              g(x + x_0) &=& (x + x_0) f(x + x_0) \\
                         &=& x f(x + x_0) + x0 f(x + x_0)
            \end{eqnarray*}
          \end{enumerate}
        \item
          $ g(x) = f(2x) $
          \begin{enumerate}
          \item
            \textbf{linear}
            \begin{eqnarray*}
              r(x) &=& \alpha f_1(x) + \beta f_2(x) \\
              H[r(x)] &=& r(2x) \\
                      &=& \alpha f_1(2x) + \beta f_2(2x) \\
              \\
              \alpha H[f_1(x)] + \beta H[f_2(x)]
                      &=& \alpha f_1(2x) + \beta f_2(2x)
            \end{eqnarray*}
          \item
            \textbf{NOT spatially invariant}
            \begin{eqnarray*}
              r(x) &=& f(x + x_0) \\
              H[r(x)] &=& r(2x) \\
                      &=& f(2x + x_0) \\
              \\
              g(x + x_0) &=& f(2(x + x_0)) \\
                         &=& f(2x + 2x_0)
            \end{eqnarray*}
          \end{enumerate}
      \end{enumerate}
  \end{enumerate}

  \newpage
  \textbf{MATLAB EXERCISES}
  \begin{enumerate}
    \item
      ------ BEGIN CODE ------
      \begin{verbatim}
% Author: <njoffe@ucsd.edu> (Nitay Joffe)
% Date: 10/03/2006
% Class: CSE 166 - Image Processing
% Homework: 1
% Problem: 1 - Working with plots and images in Matlab.

% Part (a).
% Use imread to load in the image.
image_uint8 = imread('Fig1.14(c).jpg');

% Convert the image from uint8 to double.
image_double = double(image_uint8);

% Part (b).
% Display figure as it appears in the book, using proper aspect ratio.
imshow(image_double);
truesize;

% Interactively select a ractange roughly enclosing the bottom half of the
% middle bottle.
rectangle = getrect;

% Crop the image using selected rectangle
cropped_image = imcrop(image_double, rectangle);

% Use subplot to display the cropped image next to the original image.
subplot(1,2,1);
imshow(image_double);
subplot(1,2,2);
imshow(cropped_image);

% Display the cropped image using a gray colormap and colorbar, with maximum and
% minimum gray levels corresponding to the maximum and minimum brightness in the
% image.
max_brightness = max(max(cropped_image));
min_brightness = min(min(cropped_image));
colormap('gray');
colorbar;

% Put the titles 'original image' and 'cropped image' on the respective images.
subplot(1,2,1);
imshow(image_double);
title('original image');
subplot(1,2,2);
imshow(cropped_image);
title('cropped image');
colormap('gray');
colorbar;

% Select a point on the original image that corresponds roughly to the center of
% the cropped image using ginput.
center_of_cropped_image = ginput(1);

% Plot a large white '+' at the location of the point returned by ginput.
hold;
plot(center_of_cropped_image(1), center_of_cropped_image(2), 'w+');

% Add a silly caption to the image
text(100,100,'A silly caption'); \end{verbatim}
      ------ END CODE ------ \\
    \item
      ------ BEGIN CODE ------
      \begin{verbatim}
% Author: <njoffe@ucsd.edu> (Nitay Joffe)
% Date: 10/03/2006
% Class: CSE 166 - Image Processing
% Homework: 1
% Problem: 2 - Moving Averages

% Part (a).
h_box = [1 1 1]/3;

% Create a vector x of length 100 representing a step edge.
x = [zeros(1,50) ones(1,50)];

% Add Gaussian noise with standard deviation 0.05.
x = x + 0.05*randn(1,100);

% Convolve x with h_box.
x_h_box = conv(x,h_box);

% Part (c).
% Set h_pasc equal to the fifth row  of Pascal's triangle.
h_pasc = [1 4 6 4 1];

% Normalize it so that it sums to 1.
h_pasc = h_pasc ./ sum(h_pasc);

x_h_pasc = conv(x,h_pasc);

% Part (d)
% Set x2 equal to row 375 of Figure 3.35(a).
figure_335a = double(imread('Fig3.35(a).jpg'));
x2 = figure_335a(375);

% Convolve this signal with each of hte above kernels and plot the results.
x2_h_box = conv(x2,h_box);
x2_h_pasc = conv(x2,h_pasc);

% Plot results on one page.
subplot(4,2,1);
stem(x);
title('step edge');

subplot(4,2,2);
stem(x2);
title('row 375 of figure 3.35(a)');

subplot(4,2,3);
stem(x_h_box);
title('step edge after convolution with box filter');

subplot(4,2,4);
stem(x2_h_box);
title('row 375 of figure 3.35(a) convolved with box filter');

subplot(4,2,5);
stem(x_h_pasc);
title('step edge after convolution with binomial lowpass filter');

subplot(4,2,6);
stem(x2_h_pasc);
title('row 375 of figure 3.35(a) convolved with binomial lowpass filter');

subplot(4,2,7);
stem(h_box);
title('box filter');

subplot(4,2,8);
stem(h_pasc);
title('binomial lowpass filter'); \end{verbatim}
      ------ END CODE ------ \\
      \begin{enumerate}
        \item[(b)]
          This kernel computes the average of the pixel with its direct
          neighboars. \\
          $$ y(k) = \frac{x(k-1) + x(k) + x(k+1)}{3} $$ \\
        \item[(c)]
          The binomial lowpass filter is better at smoothing out noise, but it
          also spreads out edges more. The box filter does not smooth as well,
          but edges are not as spread out when using it. To decide which to
          use, we would have to analyze how much noise there is in the signal
          and how salient the edges are and reach an optimal solution. \\
      \end{enumerate}
    \item
      ------ BEGIN CODE ------
      \begin{verbatim}
% Author: <njoffe@ucsd.edu> (Nitay Joffe)
% Date: 10/03/2006
% Class: CSE 166 - Image Processing
% Homework: 1
% Problem: 3 - Moving Differences

% Part (c).
% Centered first difference kernel.
h_cfd = [1 0 -1] / 2;

% Grab row 238 of Figure 3.35(a), in double format.
x = double(imread('Fig3.35(a).jpg'));
x = x(238,:);

% Convolve row 238 with centered first difference.
y = conv(x, h_cfd);

% Crop the filtered signal by removing its first and last samples.
y = y(2:end-1);

% Plot results.
subplot(2,1,1);
stem(x);
title('row 238 of Figure 3.35(a)');

subplot(2,1,2);
stem(y);
title('convolved with centered first difference'); \end{verbatim}
      ------ END CODE ------ \\
      \begin{enumerate}
        \item
          centered first difference ([1 0 -1]/2):
            $$ y(k) = \frac{x(k+1) - x(k-1)}{2} $$
          first difference ([1 -1]):
            $$ y(k) = x(k) - x(k-1) $$
          The centered first difference is more commonly used because it
          computes the derivative using the pixels around a given a point,
          rather than focusing only on what happens right before. This allows
          it to capture slope information both before and after a point. \\
        \item
          $ \displaystyle f'(x) = \lim_{\epsilon \to 0}
            \frac{f(x + \epsilon) - f(x)}{\epsilon} \; $
          is equivalent to the first difference. \\ \\
          $ \displaystyle f'(x) = \lim_{\epsilon \to 0}
            \frac{f(x + \epsilon) - f(x - \epsilon)}{2 \epsilon} \; $
          is equivalent to the centered first difference. \\ \\
          When $ f(x) $ comes from an image, $ \epsilon $ represents the
          distance between pixels. \\
        \item
          The positions of the spikes represent the locations where a change in
          pixel value occurs, i.e. an edge. A positive spike means the change
          is from a smaller (darker) value to a higher (brighter) value, and a
          negative spike is the reverse. The heights of the spike tell us how
          sharp of an edge, that is how much change occurs. The higher the
          spike, the more visible of a difference there is between the two
          pixels. \\
        \item
          $$ y(k) = \frac{x(k-2)}{4} - \frac{x(k)}{2} + \frac{x(k+2)}{4} $$
          To obtain this kernel from the centered first difference kernel we
          convolved the centered first difference with itself:
          [1 0 -1]/2 * [1 0 -1]/2 $ \to $ [0.25 0 -0.5 0 0.25]. \\
      \end{enumerate}
    \item
      ------ BEGIN CODE ------
      \begin{verbatim}
% Author: <njoffe@ucsd.edu> (Nitay Joffe)
% Date: 10/03/2006
% Class: CSE 166 - Image Processing
% Homework: 1
% Problem: 4 - Sampling and Aliasing

% Part (a).
x_interval = 0:16;
x_limits = [x_interval(1) x_interval(end)];
k_interval = 0:16;
N = 16;

for k = k_interval
  omega_0 = 2*pi*k/N;
  % Create cos(2*pi*k*x/N) function.
  function_handle = @(x)cos(omega_0 * x);
  % Sample values of function at x interval.
  f_n = function_handle(x_interval);

  subplot(5,4,k+1);
  % Plot continuous function.
  fplot(function_handle,x_limits);
  % Plot samples of function.  We have to specify the x interval to stem because
  % otherwise it assumes x starts at one and we need it to start at zero.
  hold;
  stem(x_interval,f_n);
  title(['k = ' int2str(k)]);
end \end{verbatim}
      ------ END CODE ------ \\
      \begin{enumerate}
        \item[(b)]
          $ f(x) $ hits the Nyquist frequency when $ k = 8 $, and $ \omega_0 =
          \pi $. \\
        \item[(c)]
          alias: A false name used to conceal one's identity. \\
          This term is used to describe $ f(n) $ when $ \omega_0 $ exceeds the
          Nyquist frequency because the sampled signal no longer appears to
          come from the continuous one. This is easiest to see in the cases
          where k=14, k=15, and k=16. If one were to try and reconstruct the
          original continuous signal from the samples, the result would be
          completely different from the original signal.
      \end{enumerate}
  \end{enumerate}
\end{document}
