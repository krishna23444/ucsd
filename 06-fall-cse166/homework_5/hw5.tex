\documentclass{article}
\usepackage{bbm}

% Margin on left side of page is 1.0 inches plus \oddsidemargin.
\oddsidemargin  -0.75in
\evensidemargin  0pt
% Margin on top also appears to be 1.0 inches plus \topmargin (not checked).
\topmargin -0.5in
\headheight 0.0in
\headsep 0.0in
\marginparwidth 0pt
\marginparsep 0pt
\textwidth   7.5in
\textheight  10.6in

\title{CSE166 - Image Processing - Homework 5}
\author{Nitay Joffe}
\date{November 14, 2006}

\begin{document}
\maketitle

\noindent
{\bf Written exercises}
\begin{enumerate}
  \item GW, Problem 8.1.
  \begin{enumerate}
    \item Can variable-length coding procedures be used to compress a histogram
          equalized image with $2^n$ gray levels? Explain.\\
          \linebreak
          No, since each $2^n$ gray level occurs with equal probability, all
          the symbols will have the same bit length $n$, and therefore no actual
          compression will be done.\\
    \item Can such an image contain interpixel redundancies that could be
          exploited for data compression?\\
          \linebreak
          Yes, there could be clusters of pixels with the same graylevels.\\
  \end{enumerate}

  \item GW, Problem 8.12.
  \begin{enumerate}
    \item How many unique Huffman codes are there for a three-symbol source?\\
          \linebreak
          1\\
    \item Construct them.\\
          \begin{itemize}
            \item 0, 10, 11\\
          \end{itemize}
  \end{enumerate}

  \item GW, Problem 8.14.\\
        The arithmetic decoding process is the reverse of the encoding
        procedure. Decode the message 0.23355 given the coding model.\\
  \linebreak
  \begin{tabular}{ccc}
    \hline
    Symbol & Probability & Range\\
    \hline
    a & 0.2 & [0.0,0.2)\\
    e & 0.3 & [0.2,0.5)\\
    i & 0.1 & [0.5,0.6)\\
    o & 0.2 & [0.6,0.8)\\
    u & 0.1 & [0.8,0.9)\\
    ! & 0.1 & [0.9,1.0)\\
    \hline\\
  \end{tabular}\\
  0.23355 $\rightarrow$ eaii!\\

  \item Consider the symmetric $2\times 2$ matrix
        $$A=\left[\begin{array}{cc}a&b\\b&c\end{array}\right].$$
        By finding the roots of the characteristic equation, 
        $$\det(\lambda I-A)=0,$$ show that the eigenvalues of A are given by
        $$\lambda = \frac{\mbox{tr}(A)\pm\sqrt{\mbox{tr}(A)^2-4\det(A)}}{2}$$\\
        \linebreak
        $$det(\lambda I-A) = \left|\begin{array}{cc}
                                     \lambda-a & b\\
                                     c & \lambda-d
                                   \end{array}\right|
          = (\lambda-a)(\lambda-d)-bc = \lambda^2-\lambda(a+d)+ad-bc$$
        $$\lambda = \frac{a+d\pm\sqrt{(a+d)^2-4(1)(ad-bc)}}{2}$$
        $$tr(A) = trace(A) = a + d$$
        $$det(A) = \left|\begin{array}{cc}
                           a & b\\
                           c & d
                         \end{array}\right| = ad-bc$$
        $$\lambda = \frac{a+d\pm\sqrt{(a+d)^2-4(1)(ad-bc)}}{2}
          = \frac{tr(A)\pm\sqrt{tr(A)^2-4det(A)}}{2}$$\\
\end{enumerate}
\end{document}
