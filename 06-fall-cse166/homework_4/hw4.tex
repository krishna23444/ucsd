\documentclass{article}
\usepackage{bbm}

\marginparwidth 0pt
\oddsidemargin 0pt
\evensidemargin 0pt
\marginparsep 0pt
\topmargin 0pt
\textwidth 6.5in
\textheight 9.0in

\author{Nitay Joffe}
\title{CSE166 Homework 4}
\date{10/24/2006}

\begin{document}
\maketitle

\noindent
{\bf Written exercises}
\begin{enumerate}
  \item \textit{GW, Problem 6.4.}\\
  We create three color filters that are closely related to the wavelengths of
  the colors of the three objects. For each filter, only the colors that
  resemble it closely will produce a high effect in the camera, while others
  will have low ones. A filter wheel can be put in to control the position of
  the filter at all times. The color white responses to all the three filters
  equally with high values, while the color black responds equally to al three
  filters with low values.
  \item \textit{GW, Problem 6.5.}\\
  R = 0.5, G = 1.0, B = 0.5. The color is teal.\\
  \item \textit{GW, Problem 6.6.}\\
  \bigskip \bigskip \bigskip \bigskip \bigskip \bigskip \bigskip \bigskip 
  \bigskip \bigskip \bigskip \bigskip \bigskip \bigskip \bigskip \bigskip
  \item \textit{GW, Problem 6.7.}\\
  Gray occurs when R = G = B, so there are $2^8$, or 256, shades of gray.\\
  \item \textit{GW, Problem 6.12.}\\
  \bigskip \bigskip \bigskip \bigskip \bigskip \bigskip \bigskip \bigskip
  \bigskip \bigskip \bigskip \bigskip \bigskip \bigskip \bigskip \bigskip
\end{enumerate}
\end{document}